%%%%%%%%%%%%%%%%%%%%%%%%%%%%%%%%%%%%%%%%%
% Medium Length Professional CV
% LaTeX Template
% Version 2.0 (8/5/13)
%
% This template has been downloaded from:
% http://www.LaTeXTemplates.com
%
% Original author:
% Trey Hunner (http://www.treyhunner.com/)
%
% Important note:
% This template requires the resume.cls file to be in the same directory as the
% .tex file. The resume.cls file provides the resume style used for structuring the
% document.
%
%%%%%%%%%%%%%%%%%%%%%%%%%%%%%%%%%%%%%%%%%

%----------------------------------------------------------------------------------------
%	PACKAGES AND OTHER DOCUMENT CONFIGURATIONS
%----------------------------------------------------------------------------------------

\documentclass{resume} % Use the custom resume.cls style

\usepackage[left=.5in,top=.5in,right=.5in,bottom=.5in]{geometry} % Document margins
%\usepackage[left=1.in,top=1.in,right=1.in,bottom=1.in]{geometry} % Document margins
%\usepackage[left=.9in,top=.9in,right=.9in,bottom=.9in]{geometry} % Document margins

\usepackage{hyperref}

\usepackage{hyperref}
\hypersetup{
    colorlinks=true,
    linkcolor=blue,
    filecolor=blue,      
    urlcolor=blue,
}
 
\urlstyle{same}

\newcommand{\tab}[1]{\hspace{.2667\textwidth}\rlap{#1}}
\newcommand{\itab}[1]{\hspace{0pt}\rlap{#1}}
\name{Nishant Mishra} % Your name
\address{(+1)~503~332~0521 } \\ % Your address
\address{\href{https://github.com/nishantmishra99}{GitHub} \\ 
\href{mailto:nishant.mishra@berkeley.edu}{Email} } \\% Your phone number and email

\begin{document}

%----------------------------------------------------------------------------------------
%	EDUCATION SECTION
%----------------------------------------------------------------------------------------

\begin{rSection}{Education}

{\bf University of California, Berkeley} \hfill {Aug 2017 - Present} 
\\ B.A. in Physics and Astrophysics \hfill {}

%Minor in Linguistics \smallskip \\
%Member of Eta Kappa Nu \\
%Member of Upsilon Pi Epsilon \\


\end{rSection}

%----------------------------------------------------------------------------------------
%	WORK EXPERIENCE SECTION
%----------------------------------------------------------------------------------------

\begin{rSection}{Research Experience}

\begin{rSubsection}{Kavli Institute for Cosmological Physics @ University of Chicago}{}{Advisor: Prof. Nickolay Gnedin}{May 2020 - Present} \\
%Collaborator(s): Oleg Gnedin (Michigan) 
\item Developing a physically motivated deep learning model to infer underlying conditions from Lyman-$\alpha$ emission spikes during the Epoch of Re-ionization.
\item Working to analyze how early galaxies contribute to the creation of emission spikes using hydrodynamic simulations.
\end{rSubsection}

\begin{rSubsection}{Cosmic Physics Center @ Fermilab}{}{Advisor: Dr. Noah Kurinsky}{May 2020 - Aug 2020}
%Collaborator(s): Enectali Figueroa-Feliciano (Northwestern) , Tarek Saab (Florida) 

\item Worked to characterizing quantum yield of eV-threshold photon detectors for the next-gen SuperCDMS Dark Matter detector.
\item Remotely operated the NEXUS detector facility at Northwestern University to take data on the admittance of the detector at a variety of temperatures.
\item Characterized the complex impedance of the detector using a two-block thermal conductance model to improve the noise modeling.
\end{rSubsection}

\begin{rSubsection}{Berkeley Center for Cosmological Physics @ University of California, Berkeley}{}{Advisor: Prof. Uro\v{s} Seljak}{May 2019 - Present}
\item Studying the variance of Lyman-$\alpha$ forest data on different simulated cosmologies.
\item Developing non-linear machine learning framework in TensorFlow to construct Lyman-$\alpha$ data sets based on a Bayesian inference model.
\end{rSubsection}
\vspace{-.8cm}
%----------------------------------------------------------------------------------------
\begin{rSubsection}{}{}{Advisor: Dr. Emmanuel Schaan}{May 2018 - Oct 2019}
\item Studied lensing of the Cosmic Microwave Background using numerical and analytic methods (in Python) for measuring gravitational lensing effects.
\item Quantified the bias of lensed of foregrounds on CMB lensing estimation and its impact on the Simons Observatory experiment.
\item Published paper in Physical Review D.
\end{rSubsection}

%----------------------------------------------------------------------------------------
\begin{rSubsection}{Texas Tech University}{ }{Advisor: Prof. Alessandra Corsi}{Jun 2016 - Aug 2017}
\item Studied the effectiveness of source extraction pipelines for the analysis of Karl G. Jansky Very Large Array (VLA) images collected as part of the radio follow-up of gravitational wave triggers during Advanced LIGO first observing run.
\item Made use of CASA software package for radio astronomy to analyze images of radio sources.
\item Co-authored paper in Astrophysical Journal Letters.
\item Worked to develop curriculum related to research for lab portion of entry level Astronomy course at TTU.
\end{rSubsection}

\end{rSection}

%----------------------------------------------------------------------------------------
%\begin{rSection}{Other Experience}

%\begin{rSubsection}{Cascadia Meteorite Laboratory, Portland State University}{ }{Advisor: Prof. Alexander %Ruzicka}{Jan 2016 - Sep 2016}
%\item Studied nature and chemical properties of meteorites at the Cascadia Meteorite Laboratory
%\item Analyzed the mineral and chemical composition of meteorites originating from the asteroid/minor planet 4 %Vesta
%\item Used a Petrographic Microscope to image (both with polarized and unpolarized light) and qualitatively %characterize objects studied then used data to select for chemical analysis
%\item Classified two meteorites from Africa as chondrites. (See publications)
%\end{rSubsection}
%
%----------------------------------------------------------------------------------------
%\begin{rSubsection}{San Jose State University}{ }{Advisor: Prof. Carolus Boekema}{Jun 2015 - Jul 2015}
%\item Studied and performed data analysis in condensed matter.
%\item Observed at the magnetic properties of GdBCO (Gadolinium Barium Copper Oxide) using $\mu$SR %(Muon-spin-rotation) data taken at Los Alamos; µSR is an experimental technique based on implanting polarized %muons.
%\item Used data on rotation and relaxation effects of the muon spin order to map the local environment.
%\end{rSubsection}
%\end{rSection}

\newpage

%----------------------------------------------------------------------------------------
\begin{rSection}{Publications}

\textbf{N. Mishra}, E. Schaan (2019), \textit{Bias to CMB lensing from lensed foregrounds}, Physical Review D, vol. 100, no. 12. \href{https://arxiv.org/abs/1908.08057}{arXiv:1908.08057}

N. T. Palliyaguru, A. Corsi, M. M. Kasliwal, S. B. Cenko, D. A. Frail, D. A. Perley, \textbf{N. Mishra}, L. P. Singer, A. Gal-Yam, P. E. Nugent, J. A. Surace (2016),  \textit{Radio follow-up of gravitational-wave triggers during Advanced LIGO O1.} The Astrophysical Journal Letters, vol. 829, no. 2. \href{https://arxiv.org/abs/1608.06518}{arXiv:1608.06518}

%A. Ruzicka, K. Bocian, \textbf{N. Mishra} (2018), \textit{Classification for Meteorite - Northwest Africa %\href{https://www.lpi.usra.edu/meteor/metbull.php?code=66379}{11548}/\href{11549}{11549}}. Meteoritical %Bulletin Database, no. 106. 

\end{rSection}


\begin{rSection}{Talks \& Presentations}
%\textbf{N. Mishra}, U. Seljak (2020, in prep) \textit{Gaussianization of Lyman-$\alpha$ Forest %power spectrum to constrain cosmological parameters}, UC Berkeley, Berkeley Physics Research %Scholars Symposium.

\textbf{N. Mishra}, N. Kurinsky (2020), \textit{Characterizing complex impedance in TES Detectors for SuperCDMS}, Fermi National Accelerator Laboratory, SULI Program Poster Session (Remote).

%\textbf{N. Mishra} (2019), \textit{Bias in CMB lensing from lensed foregrounds}, Lawrence Berkeley National %Laboratory, CMB lensing group Lunch Talk.

\textbf{N. Mishra}, E. Schaan, M. Alvarez (2018), \textit{Bias to CMB lensing from foreground lensing reconstruction}, UC Berkeley, Undergraduate Physics Symposium.

\textbf{N. Mishra} (2016), \textit{Radio follow-up of gravitation waves: Optimizing source identification in VLA images}. Texas Tech University, Clark Scholars Symposium.

\end{rSection}

\begin{rSection}{Scholarships \& Grants}

\item  - Summer Undergraduate Laboratory Internship (SULI) @ Fermilab (2020): Among 20 selected via nationwide application process, with Fermilab acceptance rate of $\sim10\%$

\item  - Berkeley Physics Research Scholar (2019-20): Stipend provided to students who demonstrate the ability and motivation to execute a research project under faculty advisor at UC Berkeley

\item  - Clark Scholar (2016): Among 12 selected via nationwide application process (500+ applicants) that provides paid research internships to HS students.




\end{rSection}
%----------------------------------------------------------------------------------------
\begin{rSection}{Relevant Courses}

\itab{\hspace{-.5cm}\textbf{Core Courses}} \tab{}  \tab{\hspace{-.5cm}\textbf{Other Courses}}
\\ \itab{\hspace{-.5cm}Analytic Mechanics (PHYS 105)} \tab{}  \tab{\hspace{-.5cm}Linear Algebra \& Differential Eq. (MATH 54)}
\\ \itab{\hspace{-.5cm}Electromagnetism \& Optics I/II (PHYS 110A/B)} \tab{}  \tab{\hspace{-.5cm}Multivariable Calculus (MATH 53)} 
\\ \itab{\hspace{-.5cm}Quantum Mechanics I/II (PHYS 137A/B)}  \tab{}  \tab{\hspace{-.5cm}Computational Techniques in Physics (PHYS 77)}
\\ \itab{\hspace{-.5cm}Statistical Mechanics (PHYS 112)} \tab{} \tab{\hspace{-.5cm}Mathematical Physics (PHYS 89)}  
\\ \itab{\hspace{-.5cm}Relativistic Astrophysics \& Cosmology (PHYS C161)} \tab{} \tab{\hspace{-.5cm}Experimental Physics I/II (PHYS 5BL/5CL)}  
\\ \itab{\hspace{-.5cm}Astrophysical Fluid Dynamics (ASTR 202)} \tab{} \tab{\hspace{-.5cm}Introductory Physics Series (PHYS 5A/B/C)} 
\\ \itab{} \tab{} \tab{\hspace{-.5cm}Introduction to Astrophysics (ASTR 7B)}
\\ \itab{} \tab{} \tab{\hspace{-.5cm}Instrumentation Laboratory (PHYS 111A)}

\end{rSection}
%----------------------------------------------------------------------------------------
\begin{rSection}{Skills}

\begin{tabular}{ @{} >{\bfseries}l @{\hspace{3ex}} l }
Programming
 &  Python (NumPy, SciPy, SymPy, Matplotlib, TensorFlow, Keras etc.)\\
 &  Java, Fortran 90, Excel, Bash/Unix, GIT, LabView, \\ \vspace{.5cm}
 &  Experience w/ NERSC/SLAC/FNAL computing resources \\

Research
 & Simulations, Machine Learning, Statistical Analysis, Technical Writing, \\
 & Basic Physics Laboratory (Circuit Design, Signal ,\\
 & Signal Generators, Oscilloscope, Arduino) \\
\end{tabular}

\end{rSection}

\newpage

\begin{rSection}{Teaching and Outreach}
\hspace{-0cm}{
\item  - SPLASH (2020): Taught a 1 hour course (An Introduction to Dark Matter Physics) to 50+ high school students (grades 9-12). Will teach the course again during future SPLASH events.

\item  - Berkeley Undergraduate Research Fair (2020): Organizing an event to connect UC Berkeley professors with undergraduates looking to do research in their labs. Also working to set up additional funding channels for undergraduate researchers including scholarships and work-study programs.

\item  - Society of Physics Students (2018-2020): Volunteered at several outreach events including Cal Day and the Bay Area Science Fest.
}

\end{rSection}

\end{document}
